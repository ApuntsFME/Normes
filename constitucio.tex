\documentclass[12pt]{article}
\usepackage[utf8]{inputenc}
\usepackage[T1]{fontenc}
\usepackage[a4paper,margin=1in]{geometry}
%\usepackage[catalan]{babel}
\usepackage[explicit]{titlesec}
\usepackage{amsthm,thmtools,enumerate}

\usepackage{fourier}
\usepackage{yfonts}
\usepackage[T1]{fontenc}

\titleformat
{\section} % command
[hang] % shape
{\scshape \bfseries\large} % format
{Títol \thesection: #1} % label
{0.5ex} % sep
{} % before-code
[] % after-code

\titleformat
{name=\section, numberless} % command
[hang] % shape
{\scshape \bfseries\large} % format
{#1} % label
{0.5ex} % sep
{} % before-code
[] % after-code


\newtheoremstyle{normal}% name
{}%         Space above, empty = `usual value'
{6ex}%         Space below
{}% Body font
{}%         Indent amount (empty = no indent, \parindent = para indent)
{\bfseries}% Thm head font
{\vspace{2ex}}%        Punctuation after thm head
{\newline}% Space after thm head: \newline = linebreak
{#1 #2: #3 }%         Thm head spec

\newtheoremstyle{unnumbered}% name
{}%         Space above, empty = `usual value'
{6ex}%         Space below
{}% Body font
{}%         Indent amount (empty = no indent, \parindent = para indent)
{\bfseries}% Thm head font
{\vspace{2ex}}%        Punctuation after thm head
{\newline}% Space after thm head: \newline = linebreak
{#1 #3}%         Thm head spec

\declaretheorem[style=normal,name=Article]{art}
\declaretheorem[style=unnumbered,name=Disposició]{disp}

\title{Constitució d'Apunts:FME}
\author{}
\date{}


\begin{document}

{\swabfamily \maketitle} %TODO titlepage

\emph{Aquest document és un esborrany i no tindrà cap valor legal mentre no sigui aprovat per l'Assemblea General d'ApuntsFME.}

{\swabfamily

\section*{Preàmbul}

Uns: quants: alumnes: de la Facultat de Matemàtiques: i Estadística de la Universitat Politècnica de Catalunya, considerant que:

És: menester que tot alumne d'aquesta facultat tingui uns: apunts: en condicions: de totes: les a"signatures:.

Malgrat que normalment es: pot obtindre tota la informació nece"sària per aprovar un examen a"sistint a cla"se, aquest mètode a vegades: és: subòptim.

La socialització dels: apunts: garanteix la llibertat de cada alumne de decidir si pren apunts: a cla"se, no els: pren o directament no se'l veu mai per la FME si no és: jugant a la botifarra.

Hi ha gent a la FME que té ma"sa temps: lliure i l'ha de dedicar a quelcom útil, i si no a quelcom no tan útil.

Decideixen constituir l'Organització ApuntsFME i per regular el seu funcionament aproven la següent

\center{\swabfamily\Large\bfseries{CONSTITUCIÓ}}

}

\section{ApuntsFME}

\begin{art}[ApuntsFME] %TODO
    \begin{enumerate}[1.]
        \item[]
        \item ApuntsFME és un Estat socialista independent que representa els interessos de tot el poble de la FME.
        \item ApuntsFME és un Estat revolucionari que ha heretat les brillants tradicions que es van formar durant la gloriosa lluita revolucionària contra els agressors imperialistes i en la lluita per aconseguir l'alliberament de la pàtria i la llibertat i el benestar del poble.
        \item ApuntsFME es guia en les seves activitats per la idea de Juche i la idea de Songun, una visió mundial centrada en les persones, una ideologia revolucionària per aconseguir la independència de les masses populars.
    \end{enumerate}
\end{art}


\begin{art}[Objectius de l'Organització] %TODO
    Els objectius de l'Organització són:
    \begin{enumerate}[a)]
        \item Per protegir el món de la devastació
        \item Per unir tots els pobles en una sola nació
        \item Per denunciar els enemics de la veritat i l'amor
        \item Per estendre el nostre poder més enllà de l'espai exterior
    \end{enumerate}
\end{art}

\section{Els integrants de l'Organització}

\begin{art}[Membres de l'Organització]
    \begin{enumerate}[1.]
        \item[]
        \item Gaudiran de la condició de Membres de l'Organització les persones que hagin realitzat una quantitat raonable de treball en algun projecte i així sigui considerat per l'Assemblea General
        \item Per acord de majoria absoluta de l'Assemblea General es podrà concedir excepcionalment la condició de Membre.
        \item Els Membres de l'Organització han de ser en tot cas estudiants matriculats a la Facultat de Matemàtiques i Estadística de la Universitat Politècnica de Catalunya. Els Membres que deixin de ser-ho perdran automàticament la condició de membres.
    \end{enumerate}
\end{art}

\begin{art}[Drets dels Membres]
    \begin{enumerate}[1.]
        \item[]
        \item Tots els Membres de l'Organització gaudeixen exactament dels mateixos drets que s'esmentin en aquesta Constitució excepte si s'especifica el contrari.
        \item Es considerarà inadmissible qualsevol discriminació contra Membres de l'Organització per qualsevol motiu.
        \item La pertinença a l'organització és voluntària. Qualsevol Membre pot abandonar l'Organització per decisió pròpia, renunciant amb això a tots els seus drets.
        \item Cap Membre podrà ser expulsat ni els seus drets reduïts per cap mitjà, excepte en casos de mala conducta greu que hagin sigut determinats prèviament i explícita. L'Assemblea General decidirà, per majoria absoluta, els mecanismes amb què es determinen les sancions.
        \item Els Membres tenen dret a ser escoltats durant la presa de decisions.
        \item Els Membres tenen dret de vot i veto en l'Assemblea General.
        \item Els Membres poden unir-se lliurement els projectes, excepte si els Estatuts d'un projecte extraordinari especifiquen el contrari. Cap Membre serà forçat a unir-se a un projecte.
    \end{enumerate}
\end{art}

\begin{art}[Deures dels Membres]
    \begin{enumerate}[1.]
        \item[]
        \item ... %TODO
    \end{enumerate}
\end{art}

\begin{art}[Associats de l'Organització]
    \begin{enumerate}[1.]
        \item[]
        \item Per acord de majoria absoluta de l'Assemblea General es pot concedir excepcionalment la condició d'Associat a qualsevol persona, àdhuc si no són estudiants de la Facultat de Matemàtiques i Estadística
        \item Els Associats no tenen dret de vot ni veto a l'Assemblea General, però l'Assemblea General els pot concedir dret de vot en projectes concrets.
        \item Els Associats poden ser desvinculats de l'Organització amb acord per majoria absoluta de l'Assemblea General.
    \end{enumerate}
\end{art}

\section{L'Assemblea General}

\begin{art}[L'Assemblea General]
    \begin{enumerate}[1.]
        \item[]
        \item L'Assemblea General és l'òrgan suprem de presa de decisions de l'Organització.
        \item Formen part de l'Assemblea General tots els Membres de l'Organització i ningú més.
        \item Els Associats tenen dret d'assistència i de veu a l'Assemblea, però mai de vot ni veto.
        \item Les normes aprovades a l'Assemblea General són jeràrquicament superiors a qualsevol altra excepte a aquesta Constitució. Qualsevol norma de rang inferior s'ha de conformar a les seves normes jeràrquicament superiors. Nogensmenys, l'Assemblea General pot autoritzar excepcions a les normes aprovades. En cas de dubte sobre aquesta conformitat, la interpretació de les normes correspon a l'Assemblea General.
    \end{enumerate}
\end{art}

\begin{art}[Sessions de l'Assemblea General]
    \begin{enumerate}[1.]
        \item[]
        \item Una tercera part dels Membres de l'Organització pot convocar una sessió de l'Assemblea General.
        \item La convocatòria ha de ser comunicada per mitjans adequats a tots els Membres.
        \item La sessió només es considera constituïda si hi assisteix un quòrum de dues terceres parts dels Membres. Els Membres poden renunciar a assistir a una sessió  de l'Assemblea i no seran comptabilitzats per decidir si hi ha quòrum.
        \item Si en algun moment durant el transcurs de la sessió deixa d'haver-hi quòrum o l'Assemblea ho decideix se suspendrà la sessió. S'intentarà reprendre la sessió el més aviat possible.
    \end{enumerate}
\end{art}

\begin{art}[Mecanismes de votació]
    \begin{enumerate}[1.]
        \item[]
        \item Totes les decisions s'han d'intentar prendre per consens. Es considera que hi ha un consens si ningú s'oposa activament a la decisió, sense comptar les possibles abstencions.
        \item En cas que, després d'escoltar totes les opinions i arguments, es consideri impossible arribar a un consens, la meitat dels membres presents pot demanar una votació.
        \item Si no s'ha especificat el contrari, les votacions es fan per majoria simple, a saber, si hi ha estrictament més vots afirmatius que negatius entre els membres presents es dóna la decisió per aprovada.
        \item Si s'especifica que una decisió s'ha de prendre per majoria absoluta, cal el vot afirmatiu de la majoria dels Membres, inclosos els qui no estiguin presents a la sessió.
        \item Si hi ha més de dues opcions es permetran votacions múltiples eliminatòries. En cada ronda l'opció o opcions amb menys vots seran descartades. En qualsevol cas l'opció majoritària haurà de ser aprovada després per majora simple o, si s'escau, absoluta.
        \item En cas de veto o d'empat en la votació no s'aprovarà cap norma i se seguirà amb la norma o costum anteriors.
    \end{enumerate}
\end{art}

\begin{art}[Dret de veto]
    \begin{enumerate}[1.]
        \item[]
        \item Qualsevol membre de l'Assemblea General té dret a vetar-ne raonadament qualsevol decisió.
        \item El dret a veto s'ha d'utilitzar de manera raonable i responsable. Per vetllar pel compliment d'aquest article, un altre membre, no necessàriament d'acord amb el veto, l'ha de considerar raonable.
        \item Sempre es consideraran raonables els vetos:
            \begin{enumerate}[a)]
                \item A una modificació de la Constitució.
                \item Si no s'han respectat els drets especificats als articles (Drets dels Membres) i (Associats de l'Organització).3 %TODO
            \end{enumerate}
        \item Els vetos seran considerats nuls i sense cap valor en els següents casos:
            \begin{enumerate}[a)]
                \item Si no s'han justificat.
                \item ... %TODO
            \end{enumerate}
    \end{enumerate}
\end{art}


\section{Els Projectes}

\begin{art}[Els Projectes]
    L'Organització s'organitza internament en Projectes. Es mantindrà un llistat actualitzat de tots els projectes.
\end{art}


\begin{art}[La creació de Projectes]
    \begin{enumerate}[1.]
        \item[]
        \item L'Assemblea General, per iniciativa d'almenys dos Membres o un Membre i un Associat, pot decidir la creació d'un projecte.
        \item El projecte ha de tenir un nom i uns objectius determinats que seran especificats en el moment de la seva creació. També s'especificarà si el projecte és temporal.
        \item Hi ha dos tipus de projecte, els ordinaris i els extraordinaris. El tipus de projecte serà decidit en el moment de la seva creació i podrà ser modificat posteriorment.
    \end{enumerate}
\end{art}



\begin{art}[Els Projectes ordinaris]
    \begin{enumerate}[1.]
        \item[]
        \item Els projectes ordinaris estan formats per uns Col·laboradors. Els Col·laboradors del projecte han de ser Membres o Associats de l'Organització. Els Associats necessiten autorització de l'Assemblea General per unir-se al projecte.
        \item L'Òrgan de decisió d'un projecte ordinari és el Consell del Projecte, formada per tots els Col·laboradors.
        \item Els mecanismes de votació del Consell del Projecte són els mateixos que els de l'Assemblea General, però els Associats no tenen dret de veto si el Consell del Projecte no els en concedeix expressament per majoria absoluta.
        \item Es poden prendre decisions del projecte sense necessitat de constituir-ne formalment una sessió del Consell, només amb la presència de dues terceres parts dels membres en el moment de prendre-les.
    \end{enumerate}
\end{art}

\begin{art}[Els Projectes extraordinaris]
    \begin{enumerate}[1.]
        \item[]
        \item Els Projectes extraordinaris tenen uns Estatuts que determinen els seus objectius i el seu funcionament intern.
        \item Els Estatuts són aprovat per l'Assemblea General i qualsevol reforma requereix la seva autorització.
    \end{enumerate}
\end{art}

\begin{art}[L'extinció dels projectes]
    \begin{enumerate}[1.]
        \item[]
        \item Un projecte es pot desactivar per consens del Consell del Projecte. El Consell sempre desactivarà un projecte una vegada completats els seus objectius.
        \item L'Assemblea General pot desactivar un projecte.
        \item Un projecte queda inactiu si cap dels seus Col·laboradors és Membre de l'Organització, excepte si l'Assemblea General autoritza la seva continuació.
        \item L'Assemblea General pot decidir la reactivació d'un projecte inactiu, amb els mateixos o diferents col·laboradors.
        \item L'Assemblea General pot decidir la dissolució definitiva d'un projecte si ha estat inactiu durant més de dos mesos o ha estat designat com a temporal en el moment de la seva creació.
    \end{enumerate}
\end{art}

\section{La reforma constitucional}

\begin{art}[Reforma constitucional]
    \begin{enumerate}[1.]
        \item[]
        \item L'Assemblea General pot, amb acord per unanimitat, aprovar una reforma constitucional.
        \item Qualsevol intent de reformar la Constitució per un procediment diferent al d'aquest article serà absolutament nul. Aquest article no pot ser modificat ni suprimit, és jeràrquicament superior a qualsevol altra norma legal i ha de ser interpretat exactament de la mateixa manera que quan va ser redactat.
    \end{enumerate}
\end{art}

\section*{Disposicions Addicionals}

\begin{disp}[addicional primera]
L'Assemblea General procedirà a crear tots els projectes necessaris per gestionar el treball ja realitzat. Fins la creació d'aquests projectes, aquest treball serà gestionat per l'Assemblea General.
\end{disp}



\section*{Disposicions Finals}

\begin{disp}[final primera]
Les persones presents a la sessió de l'Assemblea General on s'aprovi aquesta Constitució adquiriran automàticament la condició de Membres de l'Organització.
\end{disp}

\begin{disp}[final segona]
Aquesta Constitució ha de ser aprovada per unanimitat a l'Assemblea General i entrarà en vigor immediatament després de la seva aprovació.
\end{disp}

\end{document}
